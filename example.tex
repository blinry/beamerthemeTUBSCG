\documentclass{beamer}
\usetheme{TUBSCG}

\usepackage[english]{babel}
\usepackage[utf8]{inputenc}
\usepackage[T1]{fontenc}

% Show the table of contents at the beginning of each section
\AtBeginSection[]
{
    \begin{frame}<beamer>{Outline}
        \tableofcontents[currentsection]
    \end{frame}
}

%\beamerdefaultoverlayspecification{<+->} % Uncover items one by one by default
\setbeamertemplate{navigation symbols}{} % Hide the navigation symbols

\title[TUBSCG Theme]{The TUBSCG \LaTeX\ Beamer Theme}
\subtitle{Setup and Usage}
\author[S.~Morr]{Sebastian Morr}
\institute{Computer Graphics Lab\\TU Braunschweig}
\date{2013-03-01}

\begin{document}

\begin{frame}
    \titlepage
\end{frame}

\section{Setup}

\begin{frame}
    \frametitle{Prerequisites}

    You'll need:

    \begin{itemize}
        \item A \LaTeX\ distribution -- for Linux, try \alert{TeX Live}
        \item The \texttt{beamer} package. It's probably already installed
        \item Git (if you want to do the advanced setup)
    \end{itemize}
\end{frame}

\begin{frame}
    \frametitle{Dumb Setup}

    This is the dump, inflexible approach. Advantage: It's easy. Disadvantages:

    \begin{itemize}
        \item You'll duplicate the .sty each time you create a new presentation
        \item You won't benefit from any improvements of the theme on GitHub
    \end{itemize}

    Here's how to do it:

    \begin{enumerate}
        \item Download \url{https://github.com/tubs-cg/beamerthemeTUBSCG/archive/master.zip}
        \item Unzip, rename the folder to something nice
        \item Rename \texttt{example.tex} to anything you like and start editing it. You're done.
    \end{enumerate}

\end{frame}

\begin{frame}
    \frametitle{Advanced Setup}

    Here, we'll create only one copy of the .sty that is used in all your presentations.

    \begin{enumerate}
        \item Move to a directory where you want to create the new theme folder
        \item \footnotesize \texttt{git clone git://github.com/tubs-cg/beamerthemeTUBSCG} \normalsize
        \item Find the beamer package directory. Usually, it's in \texttt{[texroot]/tex/latex/beamer}
        \item Symlink \texttt{.../beamer/themes/theme/beamerthemeTUBSCG.sty} to the .sty in the newly cloned repository
        \item Run \texttt{sudo texhash}, or the equivalent on your system
    \end{enumerate}

    Now copy \texttt{example.tex} and \texttt{cg-logo.png} somewhere nice. You're done.
\end{frame}

\section{Resources}

\begin{frame}
    \frametitle{Resources}

    \begin{itemize}
        \item The \texttt{beamer} user guide \href{http://www.ctan.org/tex-archive/macros/latex/contrib/beamer/doc/beameruserguide.pdf}{\beamergotobutton{Link}}
        \item The Pro Git book \href{http://git-scm.com/book}{\beamergotobutton{Link}}
    \end{itemize}
\end{frame}

\end{document}
